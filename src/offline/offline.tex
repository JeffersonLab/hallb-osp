%--- Description of components

\infolevone{
\chapter[CLAS12 Offline Analysis]{CLAS12 Offline Analysis}
\footnote{Author: V. Ziegler \email{ziegler@jlab.org}}

The standard offline analysis software for Hall B data is ....



The XXX code performs tracking in the focal plane and reconstruction to the
target. The tracking algorithm has been shown to be accurate for events 
with one cluster per plane. Noisy events with higher cluster multiplicity
and events with more than one good track in the focal plane may not be
reconstructed correctly by the present version of the code, but work is in
progress to make this type of analysis also reliable.

The scintillator, Cherenkov, and shower counters classes perform basic decoding,
calibration (offset/pedestal subtraction, gain multiplication), and summing 
(for Cherenkovs) or cluster-finding (for showers) of hits. 
The cluster-finding algorithm of the shower class is basic and currently 
only capable of finding a single cluster per event. These classes are largely
generic and should be able to accommodate most new detectors of the respective 
type, even with a different geometry and number of channels.

Data of interest can be histogrammed and/or written to
a ROOT Tree in the output file.  The contents of the output
is defined dynamically at the beginning of the analysis.
Both 1- and 2-dimensional histograms are supported. Histograms
can be filled selectively using logical expressions (cuts).

} %end \infolevone

\infolevfour{
Table XXX lists the analysis modules available
} 
