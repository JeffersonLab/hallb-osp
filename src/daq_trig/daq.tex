\infolevone{
\chapter[Spectrometer Data Acquisition]{Spectrometer Data Acquisition
\footnote{Authors: S.Boyarinov \email{boiarino@jlab.org}}
}


\par
The Hall B data acquisition uses CODA\cite{CODAwww}
(CEBAF Online Data Acquisition), a toolkit
developed at Jefferson Lab by the Data
Acquisition Group. 
%For general information
%about CODA, see the 
%CODA site\htmladdnormallinkfoot{}{\url{http://coda.jlab.org/}}.
Up to date information about the Hall B DAQ
is kept at 
\htmladdnormallinkfoot{}{\url{http://clasweb.jlab.org/equipment/daq/daq_trig.html}}.

\par
We typically run with ......

\par
The trigger supervisor is a custom--made
module built by the data
acquisition group.  Its functions are to
synchronize the readout crates, to administer
the deadtime logic of the entire system, and
to prescale various trigger inputs.  
We have a trigger supervisor...

\par

The trigger management software is  described in
the Trigger chapter. 


\infolevtwo{

\section{General Computer Information}

\par
In the counting room we have various computers 
for DAQ, analysis, and controls.  The controls
subnet is the responsibility of .....

\section{DAQ checklist}
Things to check before experiment starts
\begin{enumerate}
\item .......
\item raster cabling
\item check portservers
\item check reset
\item check trigger latch is connected
\item helicity if needed
\item scalers
\item synchronization time stamp
\end{enumerate}

\section{Beginning of Experiment Checkout}

\par
This section describes the 
checkout of DAQ and trigger
needed before an experiment can start.

\begin{enumerate}
\item{First ensure that all the fastbus, VME,
CAMAC, and NIM crates are powered
on. 

\item{Make sure the HV is on for all detectors
and that the values are normal.}

\item{Start the scalers}
display following the instructions below and
check that the rates from detectors are normal.}
 
\item{Startup CODA using the directions below
and start a run.  With the trigger downloaded
and the HV on, you are taking cosmics data, typically at 
a rate of 3 Hz per spectrometer. }

\end{enumerate}

\section{Running CODA}

\par
This section describes how to run CODA for
the CLAS12 DAQ.  


\par

Here is how to start and stop a run.
Normally, when you come on shift, 
runcontrol will be running.  If not,
see the section on ``Cold Start'' below.
To start and stop runs, push the buttons
``Start Run'' and ``End Run'' in the
runcontrol GUI.   To change configurations
use the ``Run Type'' button.  If you have
been running you will first have to push the
``Abort'' button before you can change the 
run type. Typically the configurations
you want are the following.

\par 
A note about pedestal runs.  They have the exclusive
purpose of obtaining pedestals used for pedestal
suppression.  For details about what is done
and hints for getting pedestals for analysis (which
does not want the PEDRUN result), see \mycomp{/ped/README}.

}

\infolevthree{

\subsection{
 Some Frequently Asked Questions about DAQ}


\begin{itemize} 
\item{ {\it Q: Where is the data ?} \hskip 0.05in  
Use a command ....


Files are archived automatically to tape in the MSS
tape silo.  Two tape copies are made.  Data are
purged from disk automatically.  Users should
{\it never} attempt to copy, move, or erase data.}

\item{ {\it Q: How to adjust prescale factors ? } 
\hskip 0.05in
Edit the file .....
}

\item{ {\it Q: What is the deadtime ? } \hskip 0.05in
The deadtime is displayed in ....
}

\item{ {\it Q: Where are the crates ?} \hskip 0.05in
Fastbus crates ROCXXX are ...
}

\item{ {\it Q: Why is the deadtime so high ?  
(and related)} \hskip 0.1in
Search for answers among the following....
}

\end{itemize}

\subsection{ Quick Resets }

Problems with CODA can usually be solved with a simple
reset.  If not, try a Cold Start (see next section).
Do not waste an hour of beam time on resets; 
if they fail, call an expert.  
The expert claims he can restart CODA 
90\% of the time within 10 minutes.

If a ROC (ReadOut Controller, or crate)
is hung up, reboot by going the workspace
``Components'' and typing ......  If this 
doesn't work, try pressing the reset button 
which is on the ``Crate Resets'' section of the

\subsection{ Cold Start of CODA}

\par
If CODA is not running, or if it gets hung up,
you can do a cold start.  Frequently a subset of
these steps is sufficient to recover from a hangup,
but it takes some experience to realize the
minimum of steps that
are necessary, so the simplest 
thing is to do them all, which takes a few minutes.

\begin{itemize}
\item{Make sure the fastbus and VME crates are
running.  The crates are usually known by .....
 }
\item{ Start runcontrol and the other necessary
processes by typing ......}
\item{ In runcontrol,
press the ``Connect'' button.  
Wait 5 seconds and press ``Run Types''.  
After configure and before download, 
press the ``Reset'' button in the upper left corner.
Choose the run type from the dialog box
(see section on Running
CODA for descriptions of run types).}
\item{ After you configure and download the Run Type,
you can ``Start Run'' to start a new run.}


\end{itemize}


\subsection{ Recovering from a Reboot of Workstation}

If the workstation from which you are running CODA
was rebooted, here is how to recover DAQ.
Login as the relevant account, which is usually
a-onl for 1-DAQ operation. Passwords for the online
accounts should be available on a paper on the wall
in the counting room, or ask the run coordinator.
In the workspace for ``Components'' telnet into
all the ROCs.  If the x-terms windows are not 
available, type ......
}

\infolevone{

\section{Electronic Logbook and Beam Accounting}

Two tools are available for logging information
by the shift workers: \hskip 0.05in
(1) The Electronic Logbook, and
\hskip 0.05in
(2) The Hall Beam--Time Accounting Table.

\par
The electronic logbook is a web-based
repository of logbook data. 

\par 
The Hall Beam--Time Accounting Table is the mechanism
to summarize and record
how the beam time in a shift was spent.  
The shift leader is responsible for
submitting this table at the end of the shift.
When submitted, the data are 
logged in a database
and a summary is e-mailed to various people like the
run coordinators and the hall leader.
}

\infolevltone{\newpage}
\begin{safetyen}{10}{15}
\section{Authorized  Personnel} 
\end{safetyen}
The authorized personnel is shown in table \ref{tab:daq:personnel}.
\begin{namestab}{tab:daq:personnel}{DAQ: authorized personnel}{%
      DAQ: authorized personnel.}
  \SergeiBoiarinov{\em Contact, first on call}
\end{namestab}

\infolevthree{
\begin{table}[htp]
\centering
\begin{tabular}{|l|l|l|}  \hline
server IP &  Port & Device \\ \hline
\hline
\end{tabular}
\caption[Data Acquisition: Port Servers for DAQ]{
Port Servers for DAQ}
\label{tab:daq:portservers}
\end{table}
} %infolev
} 
