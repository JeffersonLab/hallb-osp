
%
% LaTeX Version 12 GeV ESAD 
%
% Update 7 March 2014
%

\chapter{Introduction}


This ESAD document describes identified hazards of the HPS experiment and the measures taken to eliminate, control, or mitigate them.
This document is part of the CEBAF experiment review process as defined in
\href{http://www.jlab.org/ehs/ehsmanual/manual/3120.html}{Chapter 3120 of the Jefferson Lab EHS\&Q manual},
and will start by describing general types of hazards that might be present in any of the  
JLab experimental halls.  The document then addresses the hazards associated 
with all the experimental sub-systems of Experimental Hall B and HPS and their 
mitigation.  Responsible personnel for each item is also noted.  
In case of life threatening 
emergencies call 911 and then notify the guard house at 5822 so that the guards can help
the responders.  This document does not attempt to describe the function 
or operation of the various sub-systems. Such information can be found in
the experimental hall specific Operating Manuals.

%{\it{{\bf{TO DO LIST}}
%\begin{itemize}
%\item update the outline in Chapter 3120 to match our document final ESAD
%\item review as well as remove
%the es\&h coordinators as the physics division liason does those tasks
%\item use "responsible personnel' notation through-out
%\item we can add a reference list of names at the end; but for now I have them all with
%      the sections they go with
%\end{itemize}
%}}

\chapter{General Hazards}

\section{Radiation}
	
CEBAF's high intensity and high energy electron beam is a potentially lethal direct radiation source. 
It can also create radioactive materials that are hazardous even  after the beam has been turned off. 
There are many redundant measures aimed at preventing accidental exposure to personnel by the beam 
or exposure to beam-associated radiation sources that are in place at JLab. The training and mitigation 
procedures are handled through the JLab Radiation Control Department (RadCon). The radiation safety 
department at JLab can be contacted as follows: For routine support and surveys, or for emergencies 
after-hours, call the RadCon Cell phone at 876-1743. For escalation of effort, or for emergencies, 
the RadCon manager (Vashek Vylet) can be reached as follows: Office: 269-7551, Cell: 218-2733 or Home: 772-6098.

Radiation damage to materials and electronics is mainly determined by the neutron 
dose (photon dose typically causes parity errors and it is easier to shield against). 
Commercial-off-the-shelf (COTS) electronics is typically robust up to neutron 
doses of about $10^{13} n/cm^2$. If the experimental equipment dose as calculated 
in the RSAD is beyond this damage threshold, the experiment needs to add 
an appendix on "Evaluation of potential radiation damage" in the experiment 
specific ESAD. There, the radiation damage dose, potential impact to equipment 
located in areas above this damage threshold as well as mitigating measures taken should be described.

\section{Fire}

	The experimental halls contain numerous combustible materials and flammable gases. 
In addition, they contain potential ignition sources, such as electrical wiring and equipment. 
General fire hazards and procedures for dealing with these are covered by JLab emergency 
management procedures. The JLab fire protection manager (Dave Kausch) can be contacted at 269-7674.

\section{Electrical Systems}

	Hazards associated with electrical systems are the most common risk in the experimental halls. 
Almost every sub-system requires AC and/or DC power. Due to the high current and/or high voltage 
requirements of many of these sub-systems they and their power supplies are potentially lethal 
electrical sources. In the case of superconducting magnets the stored energy is so large that 
an uncontrolled electrical discharge can be lethal for a period of time even after the actual 
power source has been turned off.  Anyone working on electrical power in the experimental Halls 
must comply with \href{http://www.jlab.org/ehs/ehsmanual/manual/6200.html}{Chapter 6200 of the Jefferson Lab EHS\&Q manual}
and must obtain approval of one of the responsible personnel. 
The JLab electrical safety point-of-contact (Todd Kujawa) can be reached at 269-7006.

\section{Mechanical Systems}

	There exist a variety of mechanical hazards in all experimental halls at JLab. 
Numerous electro-mechanical sub-systems are massive enough to produce potential fall 
and/or crush hazards.  In addition, heavy objects are routinely moved around within 
the experimental halls during reconfigurations for specific experiments. 

Use of ladders and scaffold must comply 
with \href{http://www.jlab.org/ehs/ehsmanual/manual/6132.html}{Chapter 6231 of the 
Jefferson Lab EHS\&Q manual}.
Use of cranes, hoists, lifts, etc. must comply with
\href{http://www.jlab.org/ehs/ehsmanual/manual/6141.html}{Chapter 6141 of the 
Jefferson Lab EHS\&Q manual}. 
Use of personal protective equipment 
to mitigate mechanical hazards, such as hard hats, safety harnesses, and safety 
shoes are mandatory when deemed necessary.
The JLab technical point-of-contact (Suresh Chandra) can be contacted at 269-7248.

\section{Strong Magnetic Fields}


	Powerful magnets exist in all JLab experimental halls. Metal objects may be attracted 
by the magnet fringe field, and become airborne, possibly injuring body parts or striking 
fragile components resulting in a cascading hazard condition. Cardiac pacemakers or other 
electronic medical devices may no longer function properly in the presence of magnetic fields. 
Metallic medical implants (non-electronic) may be adversely affected by magnetic fields. Loss of 
information from magnetic data storage devices such as tapes, disks, and credit cards may also occur. 
Contact Jennifer Williams at 269-7882, in case of questions or concerns.

\section{Cryogenic Fluids and Oxygen Deficiency Hazard}

Not applicable for this experiment.
%Cryogenic fluids and gasses are commonly used in the experimental halls at JLab. When released in an uncontrolled manner these can result in explosion, fire, cryogenic burns and the displacement of air resulting in an oxygen deficiency hazard, ODH, condition. The hazard level and associated mitigation are dependent on the sub-subsystem and cryogenic fluid. However, they are mostly associated with cryogenic superconducting magnets and cryogenic target systems. Flammable cryogenic gases used in the experimental halls include hydrogen and deuterium which are colorless, odorless gases and hence not easily detected by human senses. Hydrogen air mixtures are flammable over a large range of relative concentrations: from 4\% to 75\% H2 by volume. Non-flammable cryogenic gasses typically used include He and nitrogen.  Contact Kelly Mahoney at 269-7024 or Mathew Wright at 269-7722 in case of questions or concerns.

\section{Vacuum and Pressure Vessels}

	Vacuum and/or pressure vessels are commonly used in the experimental halls. Many 
of these have thin Aluminum or kevlar/mylar windows that are close to the entrance 
and/or exit of the vessels or beam pipes. These windows burst if punctured accidentally 
or can fail if significant over-pressure were to exist. Injury is possible if a failure 
were to occur near an individual. All work on vacuum windows in the experimental halls 
must occur under the supervision of appropriately trained JLab personnel. Specifically, 
the scattering chamber and beam line exit windows must always be leak checked before service. 
Contact Will Oren 269-7344 for vacuum and pressure vessel issues.

\section{Hazardous Materials}

	Hazardous materials in the form of solids, liquids, and gases that may harm people 
or property exist in the JLab experimental halls. The most common of these materials include 
lead, beryllium compounds, and various toxic and corrosive chemicals. 
Material Safety Data Sheets (MSDS) for hazardous materials 
in use in the Hall are available from the Hall safety warden.  These are being replaced by the new standard
Safety Data Sheets (SDS) as they become available in compliance with the new OSHA standards.    Handling of these materials 
must follow the guidelines of the EH\&S manual. Machining of lead or beryllia, that 
are highly toxic in powdered form, requires prior approval of the EH\&S staff. 
Lead Worker training is required in order to handle lead in the Hall. 
In case of questions or concerns, the JLab hazardous materials specialist (Jennifer Williams) can be contacted at 269-7882.

\section{Lasers}

Not applicable for this experiment.

%High power lasers are often used in the experimental areas for various purposes. Improperly  used lasers are potentially dangerous. Exposure to laser beams at sufficient power levels  may cause thermal and photochemical injury to the eye including retina burn and blindness.  Skin exposure to laser beams may induce pigmentation, accelerated aging, or severe skin burn.  Laser beams may also ignite combustible materials creating a fire hazard. At JLab, lasers with power  higher than 5 mW (Class IIIB) can only be operated in a controlled environment with proper eye protection  and engineering controls designed and approved for the specific laser system. Each specific laser systems  shall be operated under the supervision of a Laser System Supervisor (LSS) following the Laser  Operating Safety Procedure (LOSP) for that system approved by the Laboratory Laser Safety Officer (LSO).  The LSO (Bert Manzlak) can be reached at 269-7556.

