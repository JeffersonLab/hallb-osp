
\section{Electromagnetic Calorimeter}
\indent

The Electromagnetic Calorimeter (ECal) consists of $442$ lead-tungstate (PbWO$_4$) crystals with avalanche photodiode (APD) 
readout and amplifiers enclosed inside a temperature controlled enclosure. There are two identical ECal modules positioned above and below of the beam plane. In order to operate the calorimeter modules,  high voltage and low voltage are supplied to each channel. The high voltage is $<450$ V and $<1$ mA. The required low voltage is $\pm 5$ V for preamplifier boards. Constant temperature inside the enclosure is kept by running a coolant through the copper pipes that are integrated into the enclosure using laboratory chiller. Cooling system should provide temperature stability at the level of $1^\circ$C.

\subsection{Hazards} 
\indent

Hazards associated with this device are electrical shock or damage to the APDs if the enclosure is opened with the  HV on. There is also hazard associated with coolant leak that may damage preamplifier boards.

\subsection{Mitigations}
\indent

Whenever any work has to be done on the calorimeter,  whether it will be opened or not, HV and LV must be turned off. Turn chiller off if enclosure will be opened for maintenance. Any large (more than couple of degrees in C) must be investigated to make sure that there are no leaks.   



\begin{safetyen}{10}{15}
\subsection{Responsible  personnel} 
\end{safetyen}
The authorized personnel is shown in table \ref{tab:hps:personnel}.
\begin{namestab}{tab:hps:personnel}{HPS  electromagnetic calorimeter: responsible personnel}{%
      HPS calorimeter: authorized personnel.}
  \StepanStepanyan{\em 1st Contact}
  \RaphaelDupre{\em  2nd Contact}
\end{namestab}

