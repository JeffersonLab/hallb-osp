\infolevone{
\chapter[Overview]{Overview
\footnote{Authors: K. Livingston \email{kliv@jlab.org}}
}
\label{chap:controls}

The basic components of the system are:
\begin{itemize}
\item Input/Output Controllers (IOCs) - VME systems 
\item Operator Interfaces (OPI) - Computers capable of executing
EPICS tools to interact with the IOCs.
Some of the most used tools in Hall B are 
 StripTools allowing to monitor 
the behavior of one or more signals as a function of time. 
\end{itemize}

\infolevtwo{
\section{System's Components}
The tasks assigned to the various IOCs are,

\vspace{\parskip}

\begin{tabular}{r p{12.0cm}}
ioch-l & Beam current/position monitors.\\[0.5ex]
ioch-2 & Beam helicity asymmetry.....\\[0.5ex]
\end{tabular}
}

\infolevthree{
\section{Operating Procedures}
Log into the Hall B control system through one of
the computers....

To start any of these applications, use ......


\section{AlarmHandler}
The ``AlarmHandler'' notifies the user when either a signal being monitored
is outside some pre-defined limits or
communication with the IOC in which the signal resides has been lost.
``AlarmHandler'' will only detect an abnormal signal condition if
the signal is included in the application configuration file
and, the corresponding IOC database record is set to produce an alarm condition.

\section{StripTool}
Strip Tool plots a real-time strip chart of the values of one or more signals.
It is useful to monitor data trends.
A detail description of the options and operation
of this application can be found in the Strip Tool Users
Guide\htmladdnormallinkfoot{}{\url{http://www.aps.anl.gov/epics/extensions/StripTool/index.php}}
with one difference; the version used by Hall A does not have a ``print'' function.
To print a strip chart use the application ``Snapshot'' described below.

\section{Snapshot}
Snapshot refers to a KDE desktop application (ksnapshot) which allows to grab an image
of either the whole
screen or an individual window. The image can then be sent to a printer or stored on disk.

\section{Troubleshooting Procedures}
The status of most IOCs can be seen
by opening the ....

Rebooting of the IOCs is accomplished in several ways
depending on the specific IOC. 
} %infolev

\infolevltone{\newpage}
\begin{safetyen}{10}{15}
\subsection{Authorized  Personnel} 
\end{safetyen}
The authorized personnel is shown in table \ref{tab:ctrl:personnel}.
\begin{namestab}{tab:ctrl:personnel}{Slow controls: authorized personnel}{%
      Slow controls: authorized personnel.}
  \KenLivingston{}
\end{namestab}


}

