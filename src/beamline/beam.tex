% Operations Manual Chapter Title 
\infolevone{
\chapter[General Description]{General Description
\footnote{Authors: S.Stepanyan \email{stepanya@jlab.org}}
\section{Introduction}
\label{sec:beam-intro}
}}

% ESAD Section Title
\infoleveqnull{\section{Beamline}}

The control and measurement equipment along the Hall B beamline consists of various elements necessary to transport beam with required specifications onto the production target and the beam dump, and simultaneously measure the properties of the beam relevant to the successful implementation of the physics program in Hall B. 

The beamline in the Hall provides the interface between the CEBAF accelerator and the experimental hall. All work on the beamline must be coordinated with both physics division and accelerator division in order to ensure safe and reliable transport of the electron beam to the dump.

\infolevone{
The resolution and accuracy requirements in Hall B  are such that special attention is paid to the following:
\begin{list}{\arabic{enumi}.~}{\usecounter{enumi}\setlength{\itemsep}{-0.15cm}}
  \item Determination of the incident beam energy;
  \item Control of the beam position, direction, emittance and stability;
  \item Determination of the beam current;
  \item Determination of the beam polarization.
\end{list}

\subsection{Beam Diagnostic Elements}

These consist of beam position monitors (BPMs), beam current monitors,  wire 
scanners (superharps) and beam position monitors located  in 
front of the target. 

To determine the position and the direction of the beam on the experimental 
target point, two Beam Position Monitors (BPMs) are located at....



1. The averaged position over 0.3 seconds is logged into the EPICS~\cite{EPICSwww} database (1 
Hz updating frequency) and injected into the datastream every 3-4 seconds, 
unsynchronized but with an reference timestamp. From these values we can 
consider that we know the average position of the beam calculated in the EPICS 
coordinate system which is left handed.

\subsection{Beam Exit Channel}

After the target vacuum chamber.....


\section{ Machine/Beamline protection system}
\label{sec:beam-fsd}

The MPS~\cite{MPScebaf} system is composed of the Fast Shutdown System (FSD), Beam Loss 
Monitor (BLM), and gun control system.

The FSD system is a network of permissive signals which terminate at the 
electron gun and chopper 1. Devices connected to the 
FSD system include vacuum valves, RF systems, Beam loss systems, beam current 
monitors, beam dumps, and particular to Hall B, the .... .

The gun control system includes software program which monitors beam 
operating conditions and the state of the FSD and BLM systems. the program 
will warn the operators if a potential for beam damage exists. Potential for 
damage exists when running high average current beam, when FSD nodes are 
masked and when the beam power approaches the operating envelope limits for a 
specific beam dump.


\section{Safety Information}

}



%
% Information for the ESAD
%

The beamline in the Hall provides the interface between the CEBAF accelerator
and the experimental hall.   All work on the beamline must be coordinated 
with both physics division and accelerator division; in order to ensure
safe and reliable transport of the electron beam to the dump.

\subsection{Hazards}

Along the beamline these various hazards can be found.  These include
radiation areas, vacuum windows, high voltage, and magnetic fields.

\subsection{Mitigations}

All magnets (dipoles, quadrupoles, sextupoles, beam correctors) and beam 
diagnostic devices (BPMs, scanners, Beam Loss Monitor, viewers) necessary for 
the transport of the beam are controlled by Machine Control Center (MCC) 
through EPICS~\cite{EPICSwww}, except for special elements which are addressed in the 
subsequent sections. The detailed safety operational procedures for the Hall 
B beamline should be essentially the same as the one for the CEBAF machine 
and beamline.\\ 
  
\noindent{}Personnel who need to work near or around the beamline should keep in mind the potential hazards:
\begin{itemize}
  \item Radiation ``Hot Spots'' - marked by ARM or RadCon personnel,
  \item Vacuum in the beam line tubes and other vessels,
  \item Thin windowed vacuum enclosures (e.g. the scattering chamber),
  \item Electric power hazards in vicinity of the magnets,
  \item Magnetic field hazards in vicinity of the magnets, and
  \item Conventional hazards (fall hazard, crane hazard etc.).
\end{itemize}

These hazards are noted by signs and the most hazardous areas along the beamline are roped off to restrict access when operational (e.g. around the HPS chicane magnets). Signs are posted by RadCon for any hot spots. Survey of the beamline and around it will be performed before work is done on the beam line or around. The connection of leads to magnets have plastic covers for electrical safety. Any work around the magnets will require de-energizing the magnets, energized magnets are noted by read flashing beacons. Any work on the magnets requires the "Lock and Tag" procedures~\cite{EHScebaf} and the appropriate training, including the equipment-specific one. \\

\noindent{}Additional safety information is available in the following documents:
\begin{list}{--}{\setlength{\itemsep}{-0.15cm}}
  \item EH\&S Manual~\cite{EHScebaf};
  \item PSS Description Document~\cite{PSScebaf}
  \item Accelerator Operations Directive~\cite{AODcebaf};
\end{list}


\begin{safetyen}{10}{15}
\subsection{Responsible  Personnel} 
\end{safetyen}
The beamline requires both accelerator and physics personal to maintain and operate. It is very important that both groups stay in contact to coordinate any work on the Hall B beamline.   The authorized personnel is shown in table \ref{tab:beam:personnel}.
\begin{namestab}{tab:beam:personnel}{ Hall B beamline: authorized personnel}{%
       Hall B beamline: authorized personnel.}
  \FXGirod{\em 1st Contact}
  \StepanStepanyan{\em 2nd Contact}
  \MichaelTiefenback{\em  Contact  to Hall-B }
  \HariAreti{\em  Contact to Physics}
\end{namestab}
