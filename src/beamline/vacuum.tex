\infolevone{
\chapter[Hall B Vacuum System]{Hall B Vacuum System
\label{chap:vacuum}
}
\footnote{Authors: S. Stepanyan \email{stepanya@jlab.org}}
}

\infoleveqnull{\section{Vacuum Systems}}

The Hall B vacuum system consists of three segments, all interconnected. The beam transport line consisting of $1.5$ to $2.5$ inch beam pipes, the Hall B tagger magnet vacuum chamber, and the set of vacuum chambers through the HPS detector system. Only tagger vacuum chamber has a large window, $8$ inches over $30$ ft Kevlar-Mylar composite window. There is a $2.5$ in diameter $7$ mil Kapton window at the end of the HPS detector vacuum system, before the shielding wall, that is normally unaccessible. The vacuum in the system is provided by a set of rough, turbo, and ion pumps and it is maintained at the level of better than $10^{-5}$ Torr. 

\subsection{Hazards} 

\indent

Hazards associated with the vacuum system are due to rapid decompression in case of a window failure. Loud noise can cause hearing loss. Also, there is a hazard related to SVT coolant leakage into analyzing magnet vacuum chamber that will degrade the vacuum and may damage readout electronics if leak is extensive.

\subsection{Mitigations}

\indent

All personal working in the vicinity of the tagger vacuum chamber window are required to wear ear protection. Warning signs must be posted in that areas. For mitigation of a possible coolant leakage, cooling system will be interlocked with the beamline vacuum system and cooling system pressure gage. In an event of a leak, system will be shutdown and valves will be closed. 

\subsection{Responsible Personal}

\indent

The vacuumsystem will be maintained by the Hall B engineering group.  

 \begin{table}[!htb]
 \centering
 \begin{tabular}{|c|c|c|c|c|}
\hline
 Name&Dept.&Phone&email&Comments \\ \hline
 Tech-on-call & Hall-B&&& 1st contact  \\ \hline
 D. Tilles & Hall-B&&tilles@jlab.org&2nd contact \\ \hline
 \end{tabular}
\caption{ Personal responsible for the vacuum system.} 
\label{tb:beam}
\end{table}


