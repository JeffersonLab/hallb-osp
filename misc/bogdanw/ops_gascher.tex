\documentclass[12pt]{article} 
\usepackage{epsfig} 
\usepackage{graphicx} 
\begin{document}
\section{Gas Cherenkov Counter}
\subsection{Overview}

Cherenkov detectors are used for particle identification in physics experiments. 
They are based on the Cherenkov effect. The velocity of the light in transparent materials 
(glass, aerogel , gas,....) is smaller than in vacuum (their index of refraction is less than one). When
a high energy charged particle travels through these materials its velocity,
still limited to the velocity of light in the vacuum, can thus be larger
than the velocity of light in the material. When this condition occurs a
light is emitted by material, called the Cherenkov light. By detecting if a given particle
emits the Cherenkov light, one can detect if its velocity is larger than a 
threshold velocity depending on the material used. In this manual a gas 
Cherenkov for electron identification is described.
\\
Two similar threshold gas Cherenkov counters have been constructed as a part
of the particle identification equipment to be included in the focal plane 
detectors of the High Resolution Spectrometers (HRS) of the TJNAF experimental
Hall A. Each counter housing is made in steel with thin entry and exit
windows made of tedlar. Light weight spherical mirrors have also been built 
resulting in a very thin total thickness traversed by particles. The counter is
operated at atmospheric pressure with CO2.

This two counters have identical sections $(600 \times 2000)mm^2$ but
different thicknesses, $1000mm$ for the hadron arm and $1500mm$ for the
electron arm. These are gas Cherenkov which are use as threshold counters. 
The refraction in-dice of the gas is choose in order to give the maximum
light impulsion for electrons and to stay inefficient to other particles, 
for instance pions until they get $4.8GeV/c$ of impulsion. 
With CO2 at normal pressure, the refraction 
index  is $n=1.00041$ which give a threshold of
\begin{itemize}
\item[-] $p_{min} = 17 MeV/c $ for electrons
\item[-] $p_{min} = 4.8 GeV/c $ for pions
\end{itemize}

Each Cherenkov is made of 10 photomultiplier tubes (PMT) and 10 mirrors.
\begin{itemize}
\item[-] The PMT (type Burle 8854) have a spherical entrance window of
$129mm$ of diameter which only a spherical part of $110mm$ of diameter is
efficient to collect the light obtained after reflection on the mirrors. 
The photocathode is made of bi-alkali with a quantum efficiency of 22.5\% 
at $385nm$ and a extended response in the UV until $220 nm$.
\item[-] Each mirror has a rectangular profile $(350 \times 478) mm^2$
built in a empty sphere of interior radius (reflective face) 
of $900mm$ and thickness of $10mm$. The very light structure is built
like honey comb and is constituted as the following manner:
\begin{itemize}
\item[*] magnesium which protect aluminum
\item[*] aluminum which assure the reflectivity
\item[*] plexiglas which assure a good surface 
\item[*] a sandwich (carbon-epoxy , phenolic honey comb, carbon epoxy) which
assure the rigidity of the system.
\end{itemize}
\end{itemize}
The 10 mirrors are placed just before the output window and are grouped in
two columns of 5 mirrors. Each mirror reflect the light on a PMT placed at the
side of the box (figure \ref{gen}). The mirrors of the same column are
identical and the two
columns are almost symmetrical. Positions and angles of the PMT are not placed
regularly like for the mirrors but were adjusted by a optical study in order to
maximize the collection of light coming from the particular envelope of
particle which have to be detected with the High Resolution Spectrometer
(HRS) of the Hall A. PMT are fixed and mirrors can
be adjusted by hand.

The number of photoelectrons is not very high so that the 
distribution is a Poisson distribution,
$$P(N)=\frac{e^{- \bar N}(\bar N)^N}{N\!}$$
where $\bar N$ represent the number of photoelectrons 
$N_{\gamma e^-}$ collected after the
parcours in the radiator.\\
The inefficiency is defined by the probability to detect zero photo electron, if
we decide to put a threshold below one photo electron, and then is defined 
by the quantity
\begin{equation}
\epsilon=P(0)=e^{- \bar N}
\label{eff}
\end{equation}
This number of photoelectrons produced by Cherenkov effect by unity of
length crossed in the radiator is given by the following formula:
$$\frac{dN_{\gamma e^-}}{dL}=2\pi \alpha (sin \theta)^2 
\int_{\lambda_1}^{\lambda_2}\frac{1}{\lambda^2}\epsilon_Q(\lambda)
\Pi_i[\epsilon_i]d\lambda$$
with
\begin{itemize}
\item[-]$\alpha=1/137$ is fine structure constant
\item[-]$\lambda$ is the wave length
\item[-]$\theta=acos(1/\beta n)$ is the Cherenkov angle
\item[-]$\epsilon_Q(\lambda)$ is the quantum efficiency of the photo cathode
\item[-]$\epsilon_1(\lambda)$=efficiency of transmission of the radiator (we
take 
100\% )
\item[-]$\epsilon_2(\lambda)$=efficiency of reflectivity of the mirrors 
(a typical curve is given in paragraph \ref{refi}).
\end{itemize}
If we make the integral with the characteristics of the Cherenkov counters, we
expect to have 15 photoelectrons per meter of gas. With our apparatus the
length of gas crossed is 1.5 meters so we expect to have 23 photoelectrons.
 This theoretical value will be verified experimentally with beam and 
will permit to have an efficiency 
greater to 99.99\%.
\newpage
\subsection{Gas system}
\leftline{\bf General description}
\label{des}
The gas Cherenkov counter is an apparatus which will be filled with gas during
a long period. It is important to measure along
the time if there are no leakage of gas. An easy way to detect gas leakage is 
to record the entrance and the exit volume in a gas circulation maintained
with a constant flow. Figure \ref{circ} illustrate this principle.
A flexible vessel (symbolized by a cylinder in the figure) 
is connected to the gas exit of the Cherenkov box. This box is
 able to store up to 200 l 
of gas and will restitute it to the Cherenkov box, if needed, due to a 
rapid change of the ambient
temperature or atmospheric pressure.
 The gas flow recorder is a device which guarantee a
precision within $10^{-2}$ but below a low flux such as $15 l/h$ this 
precision goes down. \\
{\bf The minimum flux for the gas circulation of the 
 Cherenkov counter is $5 l/h$}.\\

\begin{figure}
\begin{center}
\includegraphics[width=15cm,height=16cm,clip]{chegasflow.eps}
{\linespread{1.}
\caption[Cherenkov gas flow]{Schematic description of the gas flow}
\label{down}}
\end{center}
\end{figure}

The counter is operated at atmospheric pressure with a very small
over-pressure due to the gas flow maintained to detect gas leakage. 
The over pressure inside the Cherenkov box is limited by the maximum value
of his volume which assure that the windows don't touch other plane
of detectors.\\ 
{\bf The value of the maximum over-pressure in the Cherenkov is 5.5 mbar}.\\
In the paragraph \ref{toto} detailed information are given about the
 variation of the volume with pressure.\\
{\bf A safety valve placed on the top of the
Cherenkov box, assure that the inside pressure don't go up to 6 mb}.

\leftline{\bf Gas rack description}
\label{upi}

Our gas system is divided in two parts; 
\begin{itemize}
\item[-] the upper part which is outside the Hall and stay near the gas 
bottles (near the gas shed),
\item[-] two gas racks associated to each Cherenkov counter.
\end{itemize}
The upper gas rack is the place where gas is extracted from bottles and is
 expanded from the bottle pressure (52 bar at T=15C, 74 bar at 31 C 
= critical point) to a pressure between 9 and 10 bars. The distribution to 
electron and hadron arm is made at this place with a Tee valve so that you
can open or close the gas line you want to select. The schematic view of this
part is shown  in figure \ref{up}. The numeration of following items is 
reported on the same figure.
\begin{enumerate}
\item There are \underline{two bottles of CO2 gas} 
which provide a high capacity CO2 storage. This correspond to a volume 
of $2\times 20m^3$. With the flow of 20l/h  this will give an autonomy 
of $\sim$42 days.
\item \underline{A system for automatic bottle switching with also expansion 
valve} which decrease the bottle pressure to a pressure of $\sim$ 10 bar.
\item \underline{A stop valve}
\item \underline{A Tee valve} for the distribution in the two gas lines, 
electron and hadron arm.
\end{enumerate}

\snfig{./figs/uppergas.eps}{\em The gas rack for the gas distribution in the electron and/or
the hadron Cherenkov counters.}{up}{5in}

\begin{figure}
\begin{center}
\includegraphics[width=15cm,height=16cm,clip]{uppergas.eps}
{\linespread{1.}
\caption[Cherenkov gas flow]{The gas rack for the gas distribution in the electron and/or
the hadron Cherenkov counters.}
\label{fen}}
\end{center}
\end{figure}

The gas rack which stay near the Cherenkov counter include the gas flow
 monitor and control the inside pressure of the box. The place of this rack is
just near the last stairs which go on the platform of the detector house. The
following list describe the elements of this rack which is diagrammed on
 figure \ref{down}.
\begin{enumerate}
\item \underline{An expansion valve from $\sim 10$ bar to 0.5 bar}. The
 gas flow recorders are devices which are working at this pressure.
\item \underline{A gas flow selector}. We can have two different flow
 inside the Cherenkov, a high flow for gas filling and a low flow 
during physics experiments. 
\item \underline{A pair of gas flow recorders}, recording both entrance 
and exit gas flow.
This provide a high sensitivity leak diagnostic within a few $10^{-3}$.
\item \underline{a flexible vessel} connected to the gas exit, able to 
store up to 200 l of gas and to restitute it if needed due to a rapid change 
of the ambient temperature or atmospheric pressure.
\end{enumerate}

\snfig{./figs/lowergas.eps}{\em The gas rack for gas flow tuning and the control of the 
inside pressure in the Cherenkov counter.}{down}{5in}

n the final position of the gas setup, final pipes and connections with the
 Cherenkov Box, we have no gas leakage. The total volume of gas which came in, 
recorded with the entrance gas meter, was within a few percent equal to the
volume recorded at the exit.
\leftline{\bf Filling Cherenkov counters with gas}
We decide to fill the volume five times 
in order to consider that is well full of CO2 gas. 
With the highest flow of 150 l/h we need 
\begin{itemize}
\item[-] $\sim$100 hours to fill one time the electron Cherenkov
 (V=2780 + 200 l)
\item[-] $\sim$66 hours to fill one time the hadron Cherenkov 
(V=1849 + 200 l)
\end{itemize}

\leftline{\bf Window deformation with pressure}
\label{toto}
Gas Cherenkov counters are working at the atmospheric pressure with a small
over-pressure ($\Delta P=P_{in}-P_{atm}=$few mbar). The windows are
 very thin (see paragraph \ref{cons})
and can be deformed with a small variation of the pressure.
We have study the deformation of the windows as a function of the 
small over-pressure in the Cherenkov box. The volume difference which correspond
to this deformation is given by the formula:
$$\Delta V=\frac{3}{4} \times H (f^{in} \times l^{in}+f^{out} \times l^{out})$$
where 
\begin{itemize}
\item[-] $H=2457$mm is the high of the window 
\item[-] $l^{in}=605$ mm is the width of the entrance window
\item[-] $l^{out}=838$ mm is the width of the exit window
\item[-] $f^{in}$ and $f^{out}$ are the ``arrows'' which represent the window 
deformation (see figure \ref{ar}).
\end{itemize}
\snfig{./figs/arrow.eps}{\em The volume deformation, due to the 
                 inside pressure, was measured through 
                 the measurement of the``arrows'' of each entrance} {ar}{5in}
\snfig{./figs/volume.eps}{\em Volume variation for the hadron Cherenkov 
counter in function of the pressure.}{vol}{5in}

To measure these ``arrows'', different over-pressures inside the
Cherenkov counters was applied, from $\Delta P=P_{in}-P_{atm}=$1 to 6 mbar. This
was maintained in the box without a gas flow by closing the output valve. For 
each over-pressure, window's deformation was measured 
(``arrows'' in figure \ref{ar}) and this operation was repeated several times
in order to obtain asymptotic value of the deformation. For over-pressures
 going from 1 to 6 mbar, the minimum and maximum values of the ``arrows'' 
, in millimeters, are:\\
\underline{$e^-$ Cherenkov}:\\
$$53.35 \leq f^{out} \leq 58.85 $$
$$36.95 \leq f^{in} \leq 39.45 $$
\underline{hadron Cherenkov}:\\
$$47.15 \leq f^{out} \leq  51.75$$
$$28.45 \leq f^{in} \leq 34.55 $$
The figure \ref{vol} represent the volume variation for the hadron Cherenkov 
counter in function of the pressure.\\
{\bf The windows bend is within tolerances for the maximum flow}.\\
We have made a test in order to know the {\bf maximum pressure that can 
support the windows}. The inside pressure on the Cherenkov box was 
increased until the
windows was torn. We have maintained this pressure without gas flow by closing
the output valve on the top of the Cherenkov box. {\bf The value of this 
over-pressure is $\Delta P=117 mbar$}. 

\leftline{\bf Empty bottle signal}
There are two bottles of CO2 gas which provide a high capacity CO2 
storage. This correspond to a volume 
of $2\times 20m^3$. With the flow of 20l/h  this will give an autonomy 
of $\sim$42 days. An automatic switching
bottle is installed in the upper gas rack, so that you are not obliged to 
move immediately to change the bottle. There are only two things to do:
\begin{itemize}
\item[-] to change the ``arm'' direction by hand on the upper-gas rack. The
pressure gauge under this device will go from 8 to 10 bar. 
\item[-] command a new bottle for the next time (the autonomy is 
at least 40 days during experiment when low flow is running).
\end{itemize} 

\leftline{\bf Check list before experiment}
This list is a summary of all procedures for gas filling and control 
which has to be done before an experiment.\\
\\
\underline{{\bf Upper gas rack}}(see figure \ref{up}):\\
If the gas is already in circulation you don't have to do the points 1 to 6. 
There are necessary only if you need to install a new bottle. 
Points 7 to 12 are the only
things that you have to check if gas is already installed.
\begin{enumerate}
\item close bottle and close orange button
\item open the purge valve (black button on the side)
\item open slightly the bottle. You should hear the gas outgoing from the purge
valve.
\item close the purge
\item open completely the bottle
\item open the orange valve
\item At this time the two bottles are open, the two orange valves also 
and the two purge are closed.
\item the switch valve (``piece of metal with a black button at the extremity)
 in high or low position indicate which bottle we are using
\item when a bottle is empty, the device (CENTRALE LF50)
 will automatically switch to the 
other bottle. In the pressure gauge instead of a pressure of 10 bars you will
read a lower pressure of 8 bars. To recover the pressure of 10 bars at this
point we have to change the direction of the switch valve by hand to the
direction
of the bottle which is now in operation.
\item open the right valve provide gas to the rack number 1 for electron arm
\item or open the left valve provide gas to the rack number 2 for hadron arm
\item or both
\item be sure that the valve of the Cherenkov counter you don't use is closed.
\end{enumerate}
\underline{{\bf gas rack number 1 or 2}}(see figure \ref{down}):\\
The pressure gauge at the but-tom of the rack indicate the over-pressure 
($\Delta P=P_{in}-P_{atm}$ in mbar) which is inside the Cherenkov counter. This
value should always be below 5.5 mbar during all operations (see paragraph 
\ref{des}).
\begin{enumerate}
\item tune the expansion valve at a pressure of maximum 0.5 bar
\item before manipulating the three channels valve (low flow on the left, 
closed for position up, and high flow on the right) be sure that the two valves
for the gas flux tuning are closed
\item for high flux, the maximum gas flow is 150 l/h which correspond 
to a pressure gauge of 5.4 mbar and a pressure inside the Cherenkov of 
($\Delta P=P_{in}-P_{atm}=$3.1 mbar
 (there is pressure lost in the 50 meters of pipes)
\item for low flux, the minimum is around 25 l/h,
 because the gas flow recorders become less precise below 15l/h,
 and the pressure inside the Cherenkov counter 
is $\Delta P=$1mbar
\end{enumerate}
Each time you can, when you go into the Hall A to make something in the
detector house, take the numbers from the two gas meter (in and out) of the
rack $n^0$1 for the electron and $n^0$2 for the hadron and report these
numbers in the experiment book. The table \ref{gamet} is an example of what
will be useful to control that there is no gas leakage.
\begin{table}
\begin{center}
\begin{tabular}{|c|c|c|c|c|c|c|}
\hline
date & $V_{in}$ & $V_{out}$ & gas flow l/h& $\Delta_{in}$ l/h &
 $\Delta_{out}$ l/h & $\Delta P$mbar \\
\hline
9/18/96 11h10 a.m & 52531 & 37407 & 30 & & & 1 \\ 
9/18/96 12h25 a.m & 52571 & 37447 & 30 & 40 & 40 & 1 \\
9/18/96 4h20 p.m & 52694 & 37571 & 30 & 123 & 124 & 1 \\
\hline
\end{tabular}
\label{gamet}
\caption{ {\em Table to control gas flow.}}
\end{center}
\end{table}

\newpage
\ssubsection{Construction}
\leftline{\bf Cherenkov box}
\label{cons}
Each Cherenkov is a box with only two walls and which is closed by two windows 
of different material. These windows are fixed by a set of two frames. There
are two identical boxes for the two detectors and as the Cherenkov for the 
electron arm is bigger there is a supplementary piece, call the riser, which
is fixed to the main box with a hermetic gasket.  

\snfig{./figs/boxdim.eps}{\em
              Each parts, windows raising and main box are fixed together
              with hermetic joins.} {boit}{5in}
              
   The figure \ref{boit} show how the different elements, 
frames, windows, boxes,  
 are fixed together. The material of these elements are steel for all
 elements expect the windows which are made of tedlar. The thicknesses are
 the following:
\begin{itemize}
\item[-] the walls of the boxes have a thickness of 2.5 mm
\item[-] the walls of the raising have a thickness of 1.5 mm
\item[-] the frames have a dimensions of $250 \times 500 mm^2$ and a thickness
of 2 mm.
\item[-] the windows are made of two leaves of tedlar of 37.5$\mu m$ of
thickness for a leaf (the density of the tedlar is 1.49 $g/cm^3$).
\end{itemize}

The figure \ref{boit1} is a detailed description of the raising, 
and in the figure \ref{boit2} the dimension of the boxes are given.

\leftline{\bf Mirrors}
The gas Cherenkov counter is positioned between other plane detectors, one
of the requirement is , therefore, to influence as less as possible the 
properties of the incoming particle. Multiple scattering and energy loss are 
thus minimized with the choice of materials with low atomic numbers and
minimizing the thickness, while keeping the requires rigidity of the 
whole structure. Since a paper where the details of the employed technique of
construction is in preparation, we will only summarize their main features.

The spherical mirrors have a radius of curvature of 90 cm. Their shape is 
such as that their ``shadow'' onto a planar surface, is a rectangle
of $35 \times 47.8 cm^2$ (see figure \ref{mirr}).
 The mirrors have been constructed with a rigid
backing made of a sandwich of phenolic honeycomb between two triple layers
of about $60 \mu m$ of carbon fiber mat ($180 \mu m$ on each side) glued with
epoxy resin. As a reflecting surface, $ 1mm$ thick aluminum plexiglas sheets 
have been used. \\
{\bf The final mirror has a total average thickness of about $230 mg/cm^2$ 
corresponding to about $\sim 5.5 10^{-3}$ radiation lengths}.

The mirrors are placed on two parallel rows and are suitably tilted to 
reflect the light towards the PMTs. As they have a spherical shape with radius 
of curvature $R=90 cm$, the PMTs are placed at a distance of $R/2=45 cm$ from
the mirrors, where the parallel rays of incident light on the mirrors are
approximately focused. To avoid ``dark zones'' between two adjacent mirrors,
they are independent one with respect to the other and overlap partially. 
Moreover, to avoid the possibility that inclined light rays could hit the mirror
along its side edge (loosing the photon which is absorbed before reaching the
mirror below), the cut along the edges of the mirror is suitably tilted. In
appendix A, detailed geometry of these mirrors are given.

Figure \ref{ref} shows a typical curve of reflectivity of the mirrors. This
value is around 90\% in the UV region which is the most important region
to cover because the number of photoelectrons produced by Cherenkov effect
is proportional to $1/\lambda^2$.

\snfig{./figs/reflec.eps}{\em
            Typical curve of reflectivity of the mirrors.}{ref}{5in}

\ssubsection{Photo-multipliers}
Five photo-multiplier tubes (PMT) are fixed of the two side walls. Each one
is surrounded by a high magnetic permeability shielding (mu-metal). The
fixing provides high voltage insulation between the PMT and the steel vessel.
A set of optic fibers provides light pulses to each PMT for their 
calibration.
\leftline{\bf Characteristics}
The type of the PMT is a Burle 8854 Photo-multiplier. This is a 129 mm
 (5 inch) diameter, 14 stage head on QUANTACON photomultiplier having 
a bialkali photo cathode of high quantum efficiency and an extremely
 high gain cesiated gallium phosphide first
dynode followed by high stability copper beryllium dynodes in the succeeding
stages. The high gain first dynode permits the direct observation of peaks 
corresponding to one two and three photoelectrons. 

The 8854 features high quantum efficiency (22.5\% at 385 nm), 
ultraviolet response (UV response to 220 nm). The number of photon from the 
Cherenkov light is proportional to $1/\lambda^2$ so it's important to have a
extended response in UV light. Figure \ref{ro} show a typical curve for
the quantum efficiency as a function of the wavelength.

\snfig{./figs/rho.eps}{\em
            Quantum efficiency of the photo cathode as a function of
            the}{ro}{5in}

Figure \ref{uni} shows a typical ADC pulse height spectrum 
(with High Voltage around 2000V) where the single
photo electron peak is very well separated from other sources of noise. Thus
they allow us to clearly define the electronic threshold to eliminate events
below this peak. The number of photo electron produced in the Cherenkov is 
expected to be large enough ($\sim 23\gamma e^{-}$) to put a threshold around
few photoelectrons which guarantee to get rid of the noise and have a very good
efficiency.


\leftline{\bf High Voltage adjustment}
If an electron is above the Cherenkov threshold, it will produce Cherenkov 
light in the gas. This light will be translate into an electric signal 
via a photomultiplier. Each photomultiplier will receive the light 
transmitted by his mirror. As a group of mirror can be touched by a good 
event, we expect to have sometimes a group of touched PMT. Before the 
discrimination of the signal we have to sum the signals over the touched 
PMT. In order to be able to make the sum , we have to know a relative 
calibration between the ten PMT of a counter. 

For the further analysis of experiment
which will use these Cherenkov counters, we need to know the efficiency of
these counters. From the spectrum of charge, we can find this efficiency from
the number of mean photoelectrons in this spectrum (see formula \ref{eff}). To
obtain this mean number of photoelectrons from the ADC spectrum, we need to
make an absolute calibration in energy of each PMT.

The high Voltage has to be adjust in order to have for each PMT the position
of the photo electron peak at the same position
above the pedestal. This adjustment has to verify some conditions:
\begin{itemize}
\item[-] to correspond to a High Voltage where the PMT noise is 
not high (below 2kHz)
\item[-] to correspond to the maximum value of the quantum efficiency
 of the photocathode.
\end{itemize}
If the two conditions are not possible at the same time, a compromise has
 to be found.

\leftline{\bf Total radiation length}

When a particle cross the Cherenkov detector, it will go through:
\begin{itemize}
\item[-] the entrance and exit window
$$\frac{x}{X_0}=\frac{2\times 37.5 10^{-4}}{28.7}=2.6 10^{-4}$$
\item[-] the gas medium
$$\frac{x}{X_0}=\frac{150}{18310}=8.2 10^-{3}$$
\item[-] one mirror
$$\frac{x}{X_0}=\sim 5.5 10^{-3}$$
\end{itemize}
The total radiation length for the whole Cherenkov detector is 
$\sim 1.4 10^{-2}$ which is equivalent of 5 mm of scintillator.

\ssubsection{Authorized Personnel}
The following individuals are authorized to work on Gas Cherenkov counters. 
\begin{itemize}
\item[~]Segal, Jack - x7242 
\item[~]Wojtsekhowski, Bogdan - x7191 
\end{itemize} 

\end{document




